\chapter{Durchführung}
\label{sec:durchfuerung}
%Die Durchführung sollte so geschrieben sein, dass alle notwendigen Angaben für den folgenden Ergebnisteil enthalten sind. Das beinhaltet z.B. Volumina von filtrierten Flüssigkeiten und die Angabe von Verdünnungen. Mit der Beschreibung der Durchführung sollte der Versuch nachvollziehbar und wiederholbar sein. Bei einer sehr umfänglichen Versuchsbeschreibung kann auf die Praktikumsanweisung verwiesen werden (1). Die groben Schritte sollten trotzdem hier in eigenen Worten beschrieben werden. Falls sinnvoll, kann auch eine Abbildung des Versuchsaufbaus eingefügt werden (Abb. 1). 

Der Versuch 1 teilt sich in zwei Versuchsabschnitte bei denen eine Müllprobe anhand von Chlor-, Asche- und Kohlenstoffgehalt charakterisiert wird. \\
In diesem Versuch wurde "`Müll I"' als Abfallprobe gewählt.\\ 

%Versuchaufbau für Versuchsabschnitt 1
Der erste Versuchsteil erfolgt am Arbeitsplatz des TIC-Moduls. Er beschäftigt sich lediglich mit der Kalibrierung des TIC-Moduls, sowie der Bestimmung des Carbonatgehaltes in der jeweiligen Müllprobe.\\

%Durchführung
%Versuchsabschnitt 1
In diesem Versuchsabschnitt wird ca. $\SI{1}{\gram}$ der Müllprobe im Erlenmeyerkolben abgewogen. Da die Probe "`Müll I"' sehr grobes und trockenes Material enthält, wird mit das Substrat mit destilliertem Wasser benetzt um eine bessere Durchmischbarkeit der Probe zu garantieren. In Folge dessen wird der Rührfisch zugegeben und der Erlenmeyer am Hals mit etwas Schlifffett versehen. Ist dies erfolgt so kann der Erlenmeyerkolben in das TIC-Modul eingespannt werden. Der Kolben steht dann auf einer auf \SI{80}{\degreeCelsius} vorgewärmten Rührheizplatte. Das Schlifffett sollte an diesem Punkt ein entweichen des zugeführten Trägergases über den Erlenmeyerkolbenhals verhindern. Trägergas ist in diesem Versuch Sauerstoff, welcher einen konstanten Trägerfluss von \SI{16}{\liter \per \hour} einzuhalten hat. Spätestens an diesem Punkt werden der mit dem Computer verbundene $CO_2$-Detektor und Rührer eingeschaltet.\\
Nach diesen Vorbereitungen kann die Messung mittels $CO_2$-Detektor über die Software am Computer gestartet werden. Es wird dabei den Anweisungen der Software folge geleistet und nach dem die Software die Basislinie gefunden hat, die Säure zum Austreiben der Carbonate zugesetzt. Dies erfolgt mit einer Hand-Dosier-Pumpe, welche über ein zwischengeschaltetes Ventil und einem Schlauch mit dem Erlenmeyerkolben und dem Phosphorsäurebehälter verbunden ist. Nach der Dosierung von zwei mal $\SI{5}{\milli \liter}$ Säure wird das zum Erlenmeyerkolben geöffnete Ventil sofort wieder geschlossen. Ab diesem Punkt wird nun gewartet bis die Software ihre Messung beendet und die Fläche unter dem erzeugen Graphen integriert hat. Start- und Endpunkt der Integration müssen möglicher Weise manuell korrigiert werden.\\
Diese Vorgehensweise bleibt für jegliche Müll- und Carbonatproben gleich und mit den erzeugten Ergebnissen kann dann ein Rückschluss auf den TIC-Gehalt in der jeweiligen Müllprobe gezogen werden.
In diesem Fall erfolgte lediglich eine Dreifachbestimmung der reinen Carbonatproben, das sich schon nach dem zweiten Messen eine genaue Kalibrierung errechnen ließ.\\
Im Nachhinein wird die verbliebene Phosphorsäure in der TIC-Abfallprobe mit $\SI{15}{\milli \liter}$ Natronlauge neutralisiert.
% Die Probe wird dann gewogen. Zur späteren Verwendung werden zu diesem Zeitpunkt ca. $\SI{3}{\gram}$ der originalen Abfallprobe in einen Keramiktiegel eingewogen. Zuletzt werden bis zum zweiten Praktikumstermin beide Proben innerhalb eines Keramiktiegels in den Trockenschrank  gestellt.

Es wird das Leergewicht zweier Keramiktiegel bestimmt.
Die zuvor analysierte Abfallprobe wird mitsamt der der Flüssigkeit in einen der beiden Keramiktiegel umgefüllt.  
In den anderen werden ca. $\SI{3}{\gram}$ der originalen Abfallprobe eingewogen. Beide Tiegel werden beschriftet und anschließend für 14 Tage in den Trockenschrank gestellt.
%Versuchsabschnitt 2

Der zweite Versuchsteil beginnt mit der Vorbereitung der Probenträger für das Verbrennungssystem EA4000. Die Kermikschiffchen werden einzeln gewogen, mit etwas gebranntem Kalk bedeckt um das Anhaften der Asche oder Schlacke beim Glühen zu unterbinden, erneut einzeln gewogen und im Anschluss vorsichtig entweder mit dem originalen Abfall oder mit der getrockneten Abfallprobe beladen welche aus dem ersten Versuchsteil überführt wurde. Ein Schiffchen wird nicht mit Abfall beladen und zur Ermittlung der Blindwertkorrektur nur mit dem gebrannten Kalk analysiert. Es stehen jetzt 3 fertige Probenträger zur Analyse bereit.
Zur Vorbehandlung des Abgases werden zwei Waschflaschen mit je \SI{80}{\milli\liter} einer 0,1 molaren Salpetersäure befüllt und an die Apparatur angeschlossen. Es ist auf den Dichtsitz der Dichtungen im Schraubdeckel Obacht zu geben.

Als erste Probe ist das Schiffchen mit dem gebrannten Kalk zur Blindwertkorrektur in das Probenrack einzulegen und die Messung am Computer zu starten.
Die Waschflüssigkeit muss nach diesem Durchgang nicht getauscht werden, weil dem reinen Kalk kein Schwefel oder Chlor entwichen sein kann. 
Es folgt der zweite Durchgang mit einer Abfallprobe. Nach dieser sind die Waschflaschen in ein gemeinsames Gefäß zu entleeren und das Gefäß zur weiteren Analyse beschriftet zur Seite zu stellen. Die Waschflaschen werden wie beim ersten Mal mit je \SI{80}{\milli\liter} einer 0,1 molaren Salpetersäure befüllt und wieder korrekt angeschlossen.
Die letzte Abfallprobe kann dem Verbrennungssystem jetzt zugeführt werden. Die Messung wird gestartet und deren Verlauf abgewartet. Die Waschflüssigkeit wird wiederum entnommen.
Sobald die Probenträger erkaltet sind kann das darauf befindliche Material entsorgt werden.\\
In Folge dessen erfolgt die Bestimmung über den Chlorgehalt argentometrisch aus den belasteten Waschflaschen mittels Titrationsautomaten. Dafür werden die Inhalte der Waschflaschen vermischt und \SI{80}{\milli \liter}, sprich die Hälfte der Probe, in das Vorlagegefäß des Automaten gegeben. Nach weiterer Vorbereitung des Rührers ist die Messung mittels Knopfdruck zu starten.