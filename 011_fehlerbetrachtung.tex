\chapter{Fehlerbetrachtung}

Es ist zu bezweifeln, dass Alle im Müll enthaltenen Carbonate mit der Phosphorsäure reagieren konnten. Papier ist sehr porös und saugt sich mit der sauren Lösung voll. Die Füllstoffpartikel liegen zwischen den Fasern vor und können so bis auf wenige Ausnahmen von der Phosphorsäure erreicht werden. Probleme treten auf wenn polymerbeschichtete Zelluloseerzeugnisse oder Kunststoffe zu analysieren sind. Die Carbonate in der aller äußersten Schicht können reagieren, während alle restlichen Carbonatpartikel von der Kunststoffmatrix umhüllt und damit von der umgebenden Lösung getrennt sind. Um auch diese Anteile erreichen zu können müsste der Müll kleinstmöglich zerkleinert werden. Thermoplastische Kunststoffe können nur mit großem Aufwand so klein zerteilt werden. Ihr (visko-)elastisches Verhalten ist das größte Hindernis. Sie können dadurch nur geschnitten oder geschert werden. Mühlen die das Mahlgut auf Druck belasten oder die Teilchen durch Impulse und Prall brechen sind nicht geeignet.
Alternativ zum Mahlen ließe sich vielleicht eine Veraschung bewerkstelligen, bei der die Carbonate keinen Temperaturen ausgesetzt sind, bei denen sie dissoziieren. Die Asche enthielte dann den gesamten anorganischen Kohlenstoff in leicht zu analysierender Form.

Der Stoffumsatz im Reaktor muss als nicht vollständig angenommen werden da die Messung aus Zeitgründen gestoppt wurden, als der gemessene emittierte Kohlenstoffdioxidstrom noch nicht null war. Obige Einschlüsse in inerte Matrizen begünstigen ebenso die Unvollständigkeit der Reaktion. 



Der Detektor unterliegt zufälligen Messabweichungen. Eine Fehlerklasse zum verwendeten Gerät ist nicht bekannt. 

Die Partikelgröße innerhalb der Müllprobe schwankte stark. Sichtbar war von Fasern mit vielleicht \SI{5}{\milli\meter} bis zu Staub mit etwa \SI{100}{\micro\meter} alles. Viel bedeutender ist das Verhältnis der enthaltenen Kunststoffe, dem Papier und der Pappe untereinander. Schon das Vorhandensein eines Polyvinylchloridpartikels kann den ermittelten Chlorgehalt stark vom wahren mittleren Chlorgehalt abweichen lassen.
Ähnlich verhält es sich auch für Schwefel und Kohlenstoff.


Das zugegebene destillierte Wasser könnte durch Spuren von Kalk verunreinigt gewesen sein. Die eventuell enthaltenen Carbonate hätten dann den Messwert nach oben verschoben.


Die Dichtigkeit des Apparates könnte Mangelhaft gewesen sein. Das vermögen des Behälters einen bestimmten Gasdruck über einen längeren Zeitraum zu halten wurde nicht geprüft. Ein großes Leck wäre aufgefallen. kleine Verluste sind allerdings nicht auszuschließen und hätten den bestimmten Carbonatgehalt gesenkt 

Leider ging während des Versuches die "`orginale Abfallprobe"' verloren. Die anschließend analysierte Probe wurde der Ursprünglichen jedoch genauest-möglich Nachempfunden. Die gleiche Menge Abfall wurde mit der exakt gleichen Menge Phosphorsäure versetzt und anschließend getrocknet.

Hohe Abweichung zur Kalibrierkurve. keine Dreifachbestimmung, schwer einzuschätzen. Zwischen den Müllproben liegt Blindprobe auf der Gerade

Erste Probe mehr Textilfasern, Zweite Probe mehr Papier/Pappe und (vermutlich) plastik Stückchen --> Mehr Carbonate (in Papier bis zu 30\%)--> quellen


