\chapter{Fehlerbetrachtung}
\label{sec:fehler}

Es ist zu bezweifeln, dass alle im Müll enthaltenen Carbonate mit der Phosphorsäure reagieren konnten. Papier ist sehr porös und saugt sich mit der sauren Lösung voll. Die Füllstoffpartikel liegen zwischen den Fasern vor und können so bis auf wenige Ausnahmen von der Phosphorsäure erreicht werden. Probleme treten auf wenn polymerbeschichtete Zelluloseerzeugnisse oder Kunststoffe zu analysieren sind. Die Carbonate in der aller äußersten Schicht können reagieren, während alle restlichen Carbonatpartikel von der Kunststoffmatrix umhüllt und damit von der umgebenden Lösung getrennt sind. Um auch diese Anteile erreichen zu können müsste der Müll kleinstmöglich zerkleinert werden. Thermoplastische Kunststoffe können nur mit großem Aufwand so klein zerteilt werden. Ihr (visko-)elastisches Verhalten stellt dabei die größte Herausforderung dar. Sie können dadurch nur geschnitten oder geschert werden. Mühlen die das Mahlgut auf Druck belasten oder die Teilchen durch Impulse und Prall brechen sind nicht geeignet.
Alternativ zum Mahlen ließe sich vielleicht eine Veraschung bewerkstelligen, bei der die Carbonate keinen Temperaturen ausgesetzt sind, bei denen sie dissoziieren. Die Asche enthielte dann den gesamten anorganischen Kohlenstoff in leicht zu analysierender Form.\\
Der Stoffumsatz im Reaktor muss als nicht vollständig angenommen werden da die Messung aus Zeitgründen gestoppt wurden, als der gemessene emittierte Kohlenstoffdioxidstrom noch nicht null war. Obige Einschlüsse in inerte Matrizen begünstigen ebenso die Unvollständigkeit der Reaktion. Für eine allgemeine Einschätzung der Müllprobe scheint dieses Vorgehen jedoch als ausreichend.\\
Der Detektor unterliegt zufälligen Messabweichungen. Eine Fehlerklasse zum verwendeten Gerät ist nicht bekannt.\\
Betrachtet man Partikelgröße, innerhalb der Müllprobe, schwankte diese stark. Es waren Fasern mit vielleicht \SI{5}{\milli\meter} bis zu Staub mit etwa \SI{100}{\micro\meter} alles. Viel bedeutender ist das Verhältnis der enthaltenen Kunststoffe, dem Papier, der Pappe und den Inertstoffen untereinander. Schon das Vorhandensein eines Polyvinylchloridpartikels kann den ermittelten Chlorgehalt stark vom wahren mittleren Chlorgehalt abweichen lassen. Ähnlich verhält es sich auch für Schwefel- und Kohlenstoffgehalte.\\
Das zugegebene destillierte Wasser könnte durch Spuren von Kalk verunreinigt gewesen sein. Die eventuell enthaltenen Carbonate hätten dann den Messwert nach oben verschoben.\\
Die Dichtigkeit des Apparates könnte Mangelhaft gewesen sein. Das Vermögen des Behälters einen bestimmten Gasdruck über einen längeren Zeitraum zu halten wurde nicht geprüft. Jedoch wäre ein großes Leck wäre aufgefallen und wurde demnach nicht registriert. Kleine Verluste sind allerdings nicht auszuschließen und hätten den bestimmten Carbonatgehalt gesenkt.\\
Geht man weiter auf die Messergebnisse und deren Auswertung ein, so ist gerade bei der Bestimmung des Carbonatgehaltes eine weitere Kalibrierung durch weitere Blindproben notwendig um eine genauere Kalibrierkurve zu erhalten. Gerade die Abweichungen von $\approx \SI{3}{\percent}$ können maximal als grobe Einschätzung gewertet werden und entsprechen keiner Laborqualität.\linebreak
Da während der Bestimmung des TIC die erste Müllprobe verloren ging, wurde eine zweite Müllprobe im weiteren Verlauf des Versuches analysiert. Dies zeigt sich unter Abschnitt \ref{sec:tic}. Die anschließend analysierte Probe wurde der ursprünglichen jedoch genauest-möglich nachempfunden. Eine nahezu ähnliche Menge Abfall des Mülls II wurde mit der exakt gleichen Menge Phosphorsäure versetzt und anschließend getrocknet.\linebreak
Optisch ließen sich jedoch Unterschiede in der Zusammensetzung erkennen. So erhielt die erste Probe der Müllprobe II mehr Textilfaser und die zweite Probe der Müllprobe II mehr Papier, Pappe und vermutlich auch Kunststoffstückchen.\linebreak
Das könnte ein Grund für den \SI{3}{\percent} höheren Carbonatgehalt, der zweiten Probe sein, da gerade in Papier, Pappe und Kunststoffen oft auch Carbonate als Füllstoffe zugesetzt werden. So finden sich in Papier, je nach Sorte, bis zu \SI{30}{\mpercent} Füllstoffe wie Kaolin, Kreide (Carbonate) oder Gips \cite{Wikipedia.21.11.2019} und in Kunststoffen bis zu  \SI{40}{\mpercent} Carbonate wieder \cite{PolymerServiceGmbHMerseburg.13.08.2019}, um den geforderten Eigenschaften gerecht zu werden.\\
Zu den berechneten Heiz- und Brennwerten ist noch zu erwähnen, dass diese nur mittels Näherungsformeln berechnet wurden und somit nur eine grobe Einschätzung zulassen. Jedoch ist eine genauere Berechnung für diesen Versuch nicht nötig gewesen, da der Charakter der Müllprobe II deutlich wurde und durch Inhomogenität des Mülls eine größere Genauigkeit der Heiz- bzw. Brennwertbestimmung keinen Mehrwert für diesen Versuch gibt.\\ \\
\textcolor{red}{Ließ mal nochmal drüber pls so als}\\
Abschließend lässt sich sagen, dass die Charakterisierung der Müllprobe II mit den dargelegten Messmethoden für eine grobe, qualitative Einschätzung des Mülls geeignet sind. Genauere Messmethoden könnten bei tieferer Charakterisierung nötig sein, gerade wenn es um die quantitative Zusammensetzung des Mülls geht.