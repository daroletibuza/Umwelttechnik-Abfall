\chapter{Fehlerbetrachtung}

Es ist zu bezweifeln, dass Alle im Müll enthaltenen Carbonate mit der Phosphorsäure reagieren konnten. Papier ist sehr Porös und saugt sich mit der sauren Lösung voll. Die Füllstoffpartikel liegen zwischen den Fasern vor und können so bis auf wenige Ausnahmen von der Phosphorsäure erreicht werden. Probleme treten auf wenn polymerbeschichtete Zelluloseerzeugnisse oder Kunststoffe zu analysieren sind. Die Carbonate in der aller äußersten Schicht könne reagieren, während alle restlichen Carbonatpartikel von der Kunststoffmatrix umhüllt und damit von der umgebenden Lösung getrennt sind. Um auch diese Anteile erreichen zu können müsste der Müll kleinstmöglich zerkleinert werden. Thermoplastische Kunststoffe können nur mit großem aufwand so klein zerteilt werden. Ihr (visko-)elastisches Verhalten ist das größte hinderniss. Sie können dadurch nur geschnitten oder gescheert werden. Mühlen die das Mahlgut auf Druck belasten oder die Teilchen durch Impulse und Prall brechen sind nicht geeignet.
Alternativ zum Mahlen ließe sich vielleicht eine Veraschung bewerkstelligen, bei der die Carbonate keine Temperaturen ausgesetzt sind, bei denen sie dissoziieren. Die Asche enthielte dann den gesamten inorganischen kohlenstoff in leicht zu analysierender Form.

Stoffumsatz im primitiven Reaktor

Der Stoffumsatz im Reaktor muss als nicht vollständig angenommen werden da die Messung aus Zeitgründen gestoppt wurden, als der gemessene emmittierte Kohlenstoffdixidstrom noch nicht null war. Obige Einschlüsse in inerte Matrizen begünstigen ebenso die Unvollständigkeit der Reaktion. 


Chemisches GGW? mal noch aufstellen "`K"'???

Detektor
Der Detektor unterliegt zufälligen Messabweichungen. 

Homogenität der Müllproben
Die Partikelgröße innerhalb der Müllprobe schwankte stark. Sichtbar war von Fasern mit vielleicht \SI{5}{\milli\meter} bis zu Staub mit etwa \SI{100}{\micro\meter} alles. Viel bedeutender ist das Verhältnis der enthaltenen Kunstoffe, dem Papier und der Pappe untereinander. Schon das vorhandensein eines Polyvinylchloridpartikels kann den ermittelten Chlorgehalt stark vom wahren mittleren Chlorgehalt abweichen lassen.
Ähnlich verhält es sich auch für Schwefel und Kohlenstoff.


Reinheit des zugegebenen Wassers

Gütegrad der Phosphorsäure

Aufschließbarkeit der Caronate im Müll (Als Füllmittel in Kunsstoffen sind sie für die Säure nur schwer erreichbar)

Dichtigkeit der Verbindungen(kein Druckhaltevermögen über Zeitraum getestet)

Der Verlust der Orginal Abfallprobe. Wurde aber Nachempfunden

Hohe Abweichung zur Kalibrierkurve. keine Dreifachbestimmung, schwer einzuschätzen. Zwischen den Müllproben liegt Blindprobe auf der Gerade

Erste Probe mehr Textilfasern, Zweite Probe mehr Papier/Pappe und (vermutlich) plastik Stückchen --> Mehr Carbonate (in Papier bis zu 30\%)--> quellen


