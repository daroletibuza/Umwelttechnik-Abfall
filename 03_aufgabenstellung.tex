\chapter{Aufgabenstellung}
\label{sec:aufgabenstellung}
%In der Aufgabenstellung wird (in eigenen Worten und ganzen Sätzen) formuliert, was das Ziel des 
%Versuches ist.  
%[Beachten Sie die eigentliche Aufgabenstellung in den Versuchsanleitungen sowie die Hinweise zur Auswertung!] 
Im Versuch 1 "`Thermische Abfallbehandlung"' wird eine der vorliegenden Abfallproben gewählt und nach Asche-, Chlor- und insbesondere Kohlenstoffgehalt charakterisiert. \\ \\
Der gesamte Versuch teilt sich dabei in zwei Versuchsteile. Im ersten Versuchsteil erfolgt die Kalibrierung des TIC-Moduls und die Bestimmung des Carbonatgehaltes in der Probe. Im zweiten Versuchsteil wird dann über Totalverbrennung in einer Brennkammer der thermisch vollständig zersetzte Abfall bilanziert.\\ \\
Besonders im Fokus steht in diesem Versuch die Plausibilität der Messwerte mit deren Interpreation, sowie die Fehleranfälligkeit mit Vor- und Nachteilen der einzelnen Messmethoden.
