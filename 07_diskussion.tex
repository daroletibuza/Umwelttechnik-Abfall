\chapter{Diskussion}
\label{sec:diskussion}

In diesem Abschnitt wird die Plausibilität der erhobenen Daten zur Analyse der Müllprobe 2 überprüft und diskutiert.\\ \\
Beginnend mit dem Carbonatgehalt bzw. dem TIC-Gehalt werden diese als wichtige Füll- und Zuschlagsstoffe für Papier, Pappe und diverse Kunststoffe eingesetzt. Für die untersuchte Müllprobe 2 fällt mit einem mittleren Carbonatgehalt von $\SI{8.9}{\mpercent}$ (siehe Abschnitt \ref{sec:tic}) dieser im Vergleich zu Carbonatgehalten für Papier mit bis zu $\SI{30}{\mpercent}$ \cite{Wikipedia.21.11.2019} und für Kunststoffe von $10 - \SI{50}{\mpercent}$ \cite{PolymerServiceGmbHMerseburg.13.08.2019} gering aus. Da aus der optischen Betrachtung des Mülls nicht nur Papier- und Kunststoffreste zu erkennen sind, sondern auch fasrige und \textbf{ veraschte} Komponenten, ist es durchaus plausibel, dass der Carbonatgehalt entsprechend kleiner ausfällt. \\
In diesem Versuch wurde der Carbonatgehalt für zwei Stichproben bestimmt, welche von einander eine Abweichung von 3\% aufweisen, was sich auf die inhomogene Zusammensetzung der Müllprobe selbst zurück führen lässt.\linebreak
Die Abweichungen bezüglich der Kalibrierkurve könnten sich ebenfalls auf diese Inhomogenität zurückführen lassen, jedoch wäre an dieser Stelle weitere Kalibrierpunkte, sowie weitere Messungen mit Müllproben sinnvoll, um repräsentative Aussagen treffen zu können. Nähere Infos zur Fehlerbetrachtung finden sich unter Abschnitt \ref{sec:fehler}.\\

Der TC-Gehalt liegt laut den Messergebnissen bei rund $\approx 25\%$ bzw. aus der Summe zwischen TIC und TOC bei  $\approx 23\%$ (siehe Tabelle \ref{tab:tc}). Aufgrund der optischen Zusammensetzung des Mülls teilweise enthaltenen Papier, Pappe und Kunststoffschnipseln sowie den Textilfasern und ascheähnlichen Resten könnte der Kohlenstoffgehalt plausibel sein. Verglichen mit Tabelle \ref{tab:tc_vergleich} \cite[S.13]{HansGunterRamke.} scheint sich die Plausibilität mit den Daten für Feinmüll und gemischt mit Pappe, Papier und Kunststoffen zu decken.\\
Gegenüber anderer Untersuchungen ist der in diesem Versuch ermittelte Kohlenstoffgehalt eher niedrig\cite{Schwarzboeck2018}. Die Müllprobe kommt einem Ersatzbrennstoff sehr nah. Darum die Ergebnisse mit Artikeln über selbige verglichen werden.

TABELLLLLLE von \cite{HansGunterRamke.}
Hans Gunter bezieht sich auf Kompost oder? 
\vspace*{5mm}

\begin{itemize}
	\item Plausibilität TC \cite{HansGunterRamke.}
	\item 
	\item Plausibilität TS
	Der Schwefelgehalt des Abfalls setzt sich aus dem Organischen Schwefelgehalt (TOS) und dem Anorganischen Schwefelgehalt(TIS) zusammen. Aus dem Artikel \cite{Schwarzboeck2018} lässt sich ein Referenzwert für den TOS von etwa 8 bis \SI{10}{\gram\per\kilogram} bei Asche- Und Wasserfreiheit ableiten. Das entspricht \SI{0,64}{\mpercent} bis etwa \SI{1}{\mpercent}(siehe Rechnung weiter unten vielleicht noch in den Anhang)
	Schwefelgehalte des Hausmülls schwankten mitte der 1990ger Jahre zwischen \SI{0,3}{\mpercent} und \SI{0,5}{\mpercent} \cite{scholz2013}.
	

	\item Plausibilität Wassergehalt
	Die Feuchtigkeit in der Untersuchten Müllprobe kann praktisch nur im Papier und ähnlichem enthalten sein. Kunststoffe binden Feuchtigkeit praktisch nur als Anhaftung an ihrer Oberfläche \cite{LLA_Abfallanalyse}. Der Müll fühlte sich trocken an und haftete auch nicht feuchtigkeitsbedingt zusammen. Erwartungsgemäß war der Wasseranteil mit 3,29\% auch sehr gering.
	Eine vielleicht ähnlich beschaffene Abfallprobe aus Papier, Pappe, Windeln, Textilien, Leder und Gummi enthielt \SI{64}{\kilogram\per\tonne} Wasser, was \SI{6,4}{\mpercent} Wasser entspricht \cite{scholz2013}.
	\item Plausibilität Glühverlust
	
	
	Obige Probe aus Papier, Pappe, uvm. enthielt außerdem \SI{2,1}{\mpercent} Inertstoff, also Asche \cite{scholz2013}.
	\item Plausibilität Heizwert \cite{S.Furkus.}
	
	Bereits beschriebene Probe aus Papier, Pappe, Windeln, Leder, Textilien und Gummi wies einen Heizwert von \SI{11,96}{\mega\joule\per\kilogram} auf \cite{scholz2013}.
	Das liegt weit über dem Heizwert der hier analysierten Probe. Die Zusammensetzung der beiden Proben ist nicht genauer bestimmt. Es werden lediglich Aussagen zur qualitativen Beschaffenheit getroffen. Ein Widerspruch ist hier nicht zu erkennen, da die Werte immerhin in der selben Größenordnung liegen.
	Erwartungsgemäß ist der Heizwert geringer als der Brennwert da die latente Wärme des Wasserdampfes nicht verwertet wird.
	\item Plausibilität Brennwert
	Der Brennwert ist, richtiger Weise, höher als der Brennwert da die im Wasserdampf gebundene Energie mit betrachtet wird. Ein Gemisch  welches zu gleichen Teilen aus Haus- und Papiermüll besteht hat etwa einen Heizwert von \SI{3,5}{\kilo\watt\hour\per\kilogram} oder \SI{12,6}{\mega\joule\per\kilogram} \cite{S.Furkus.}. Mit\SI{5,94}{\mega\joule\per\kilogram} ist der berechnete Brennwert etwa halb so groß. 
	
\end{itemize}

%===========================================
%Optional Berechnung zum Schwefelgehalt.

%	Eine Beladung von 8 - 10g/kg mit Schwefel bei Asche und Wasserfreiheit
%	\begin{flalign}
%	\frac{10 g}{1000 g} &= 1\%\\
%	\frac{8 g}{1000 g} &= 0,8\%
%	\end{flalign}
%	Angleichung an reale Verhältnisse durch Annahme von maximal 20 ma\% Asche und 5\% Wasser.
%	\begin{flalign}
%	\frac{10 g}{1250 g} &= 0,8\%\\
%	\frac{8 g}{1250 g} &= 0,64\%
%	\end{flalign}
%=============================================
\newpage

Die wichtigste anorganische Kohlenstoffquelle in der Müllprobe sind vermutlich Carbonate.
Carbonate sind wichtige Füll- und Zuschlagsstoffe für Papier, Pappe und diverse Kunststoffe. Herkömmliches Papier kann dabei einen Füllstoffanteil von bis zu 30 \% \cite{Wikipedia.21.11.2019} aufweisen. Im Kunststoff PVC-U dient Calciumcarbonat unter anderem zur Steigerung der Schlagzähigkeit und der Oberflächengüte. Hart-PVC profitiert durch die Anhebung des E-Moduls, der Bruchdehnung, Schlagzähigkeit, Zugfestigkeit und des Oberflächenglanzes. Selbst die Wetterbeständigkeit verbessert sich. Bei vielen weiteren Kunststoffen ergibt sich durch die Zugabe von Calciumcarbonat neben einer Vielzahl von werkstofftechnischen Verbesserungen auch ein  entscheidender Preisvorteil. \cite{domininghausKunststoffeUndIhre1998}. Der Anteil reicht dabei von 3\% bis etwa 50\% Calciumcarbonat \cite{domininghausKunststoffeUndIhre1998}.

Die unbekannte Zusammensetzung der Müllprobe und die uneinheitliche Anwendung von Zuschlag- und Füllstoffen lassen kaum einen Vergleich mit anderen Proben zu. 

In der Endkonsequenz könnte jeder Carbonatgehalt durch Füllstoffe erklärt werden. Ein weiterer Einflussfaktor sind Verunreinigungen. Teilweise werden anfallende Abfälle nicht korrekt getrennt. Ein wahrscheinliches Szenario wäre die Entsorgung nicht vollkommen restentleerter Kalksäcke. Die Säcke bestehen aus einer Kunststoffschicht und einer äußeren Papierhülle. Der enthaltene Kalk verfälscht die Messergebnisse. Ebenso könnte Kehricht im untersuchten Müll entsorgt worden sein.\\


Der Chlorgehalt könnte zwischen etwa \SI{0,3}{\mpercent} und \SI{3}{\mpercent} \cite{LLA_Abfallanalyse}.

Chlorgehalte des Hausmülls schwankten mitte der 1990er Jahre zwischen \SI{0,4}{\mpercent} und \SI{1}{\mpercent} \cite{scholz2013}.\\



Das Papier (Illustrationspapier) aus einem Magazin weist eine Glührückstand von ca. \SI{15}{\mpercent} auf \cite{roempppap}. Einige Kunststoffe können theoretisch rückstandsfrei verbrennen. Es besteht also ein enger Bezug zum Füllstoffgehalt. Dieser Füllstoffanteil wurde zuvor schon ausreichend diskutiert. Ein realistischer Mittelwert für den Glührückstand von Kunststoffen wäre wohl \SI{20}{\mpercent}.
Aschegehalte des Hausmülls schwankten mitte der 1990ger Jahre zwischen \SI{25}{\mpercent} und \SI{35}{\mpercent} \cite{scholz2013}.
Der im Experiment ermittelte Glührückstand von rund 75\% ist deutlich zu hoch. In der Literatur ist nirgends ein so hoher mineralischer Füllstoffanteil, Glührückstand oder Aschegehalt beschrieben. Es sind 5 wichtige Fehlerquellen zu beachten. Der Müll könnte durch einen beachtlichen mineralischen Feststoffanteil belastet sein. Die Veraschung könnte unvollständig sein, wodurch verbliebene Organische Anteile als anorganisch interpretiert würden. Die Asche könnte bei den angewendeten Temperaturen ihre chemische Zusammensetzung in Anwesenheit von Umgebungsluft verändert haben. Die Asche könnte Feuchtigkeit aus der Umgebungsluft aufgrund ihrer hygroskopischen Eigenschaften angezogen haben. Es könnten Messfehler passiert sein. Bei Vertauschung der Tiegel könnten die Massen falsch subtrahiert worden sein. Dann hätten die Werte aber stark von einander abweichen müssen, weil sich der Fehler für die eine Masse negativ und für die andere Masse positiv ausgewirkt hätte. Letzterer Fehler kann damit ausgeschlossen werden. 
In Anbetracht der vielen möglichen Einflussfaktoren muss der Ermittelte Wert kritisch betrachtet und vorsichtig verwendet werden. Er kann aber nicht als Falsch ausgeschlossen werden.
%\begin{tikzpicture}
%\begin{axis}[
%xbar=1pt,% space of 0pt between adjacent bars
%bar width=11,
%width=15cm,
%height=11cm,
%%minor y tick num=4,
%xmax=40,xmin=-40,
%x tick label style={/pgf/number format/.cd,%
%	scaled x ticks = false,
%	set decimal separator={,},
%	fixed},
%symbolic y coords={Zn,Cl,Mn,Uub,As},
%ytick=data,
%xtick={-30.5,-20.5,-10.5,0,10.5,20.5,30.5},
%grid=major,
%%enlargelimits=0.15,
%]
%\addplot coordinates {
%	(-10.3,Mn) (15.4,Cl) (5,Zn) (24,Uub) (30,As)
%};
%\addplot coordinates {
%	(-3,Mn) (5,Cl) (15,Zn) (20,Uub) (35,As)
%};
%\addplot coordinates {
%	(-8,Mn) (-19,Cl) (20,Zn) (30,Uub) (5,As)
%};
%
%\end{axis}
%\end{tikzpicture}


