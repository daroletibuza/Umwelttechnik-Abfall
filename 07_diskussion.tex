\chapter{Diskussion}
\label{sec:diskussion}

In diesem Abschnitt wird die Plausibilität der erhobenen Daten zur Analyse der Müllprobe I überprüft und diskutiert.\\ \\
Beginnend mit dem Carbonatgehalt werden diese als wichtige Füll- und Zuschlagsstoffe für Papier, Pappe und diverse Kunststoffe eingesetzt. Für die untersuchte Müllprobe I fällt mit einem mittleren Carbonatgehalt von $\SI{5,3}{\mpercent}$ (siehe Abschnitt \ref{sec:tic}) dieser im Vergleich zu Carbonatgehalten für Papier mit bis zu $\SI{30}{\mpercent}$ \cite{Wikipedia.21.11.2019} und für Kunststoffe von $10 - \SI{50}{\mpercent}$ \cite{PolymerServiceGmbHMerseburg.13.08.2019} gering aus. Da aus der optischen Betrachtung des Mülls nicht nur Papier- und Kunststoffreste zu erkennen sind, sondern auch fasrige und veraschte Komponenten, ist es durchaus plausibel, dass der Carbonatgehalt entsprechend kleiner ausfällt. Möglicherweise ist Kehricht, mit einem gewissen Carbonatanteil, im untersuchten Müll entsorgt worden.\\
In diesem Versuch wurde der Carbonatgehalt für zwei Stichproben bestimmt, welche von einander eine Abweichung von 3\% aufweisen, was sich auf die inhomogene Zusammensetzung der Müllprobe selbst zurückführen lässt.\linebreak
Die Abweichungen bezüglich der Kalibrierkurve könnten sich ebenfalls auf diese Inhomogenität zurückführen lassen, jedoch wäre an dieser Stelle weitere Kalibrierpunkte, sowie weitere Messungen mit Müllproben sinnvoll, um repräsentative Aussagen treffen zu können. Nähere Informationen zur Fehlerbetrachtung finden sich unter Abschnitt \ref{sec:fehler}.\\
%Plausibilität TC
Der TC-Gehalt liegt laut dem Analyse-Brennkammer bei $\approx 27\%$ (siehe \mbox{Tabelle \ref{tab:tc_messung}}). Aufgrund der optischen Zusammensetzung des Mülls mit teilweise enthaltenen Papier, Pappe und Kunststoffschnipseln sowie den Textilfasern und ascheähnlichen Resten könnte der Kohlenstoffgehalt plausibel sein. Verglichen mit Tabelle \ref{tab:tc_vergleich} \mbox{\cite[S.11]{HansGunterRamke.}} scheinen sich die Daten für Feinmüll, gemischt mit Pappe, Papier und Kunststoffen, bzw. grob aufgerundet, mit dem unteren Wert von Hausmüll mit \SI{30}{\mpercent} zu decken.\\
Für die Blindprobe und die vorbehandelte Probe durch die Bestimmung des TIC fallen andere Werte als die der originalen Müllprobe an. Dies ist, aufgrund der unterschiedlichen Zusammensetzungen aller drei Proben, auch zu erwarten. Für die Blindprobe ist ein sehr geringer Messwert, durch das Fehlen an Kohlenstoff in der Verbindung von \ce{CaO}, zu erwarten. Dies deckt sich auch mit dem ausgebenen Messwert der Analyse von \SI{0,41}{\mpercent}. Der Restkohlenstoffgehalt kann dabei auf die nicht 100\%ige Reinheit der Probe zurückgeführt werden bzw. weiterer Faktoren, welche unter Abschnitt \ref{sec:fehler} in der Fehlerbetrachtung aufgeführt sind.\\
Der Messwert für die vorbehandelte Probe durch die TIC-Bestimmung liegt bei \SI{4,39}{\mpercent}. Fälschlicherweise könnte man erwarten, dass dieser Wert, aufgrund des ausgetriebenen, anorganischen Kohlenstoffs bei $\approx \SI{26}{\mpercent}$ liegen müsste. Da jedoch für das Austreiben des \ce{CO2}, aus den Carbonaten, Phosphorsäure und zur Neutralisation Natronlauge hinzugefügt wurde, verfälschen diese die Masse der Probe und man erhält somit einen verfälschten Messwert. Möglich wäre es an dieser Stelle, die Masse der zugeführten Säure und Base einzubeziehen und somit einen neuen, hier nicht aufgeführten, Messwert zu errechnen.
%Tabelle START
\vspace*{-3.5mm}
\renewcommand{\arraystretch}{1.2}
\begin{table}[h!]
	\centering
	\caption[Elementgehalte von Restabfall und ausgewählter Fraktionen]{Elementgehalte von Restabfall und ausgewählter Fraktionen \cite[S.11]{HansGunterRamke.}}
	\label{tab:tc_vergleich}
	\resizebox{15cm}{!}{
	\begin{tabulary}{1.2\textwidth}{L|CCCCC}
		\hline
		\textbf{Parameter}	& \mbox{\textbf{Hausmüll \, \,}} & \textbf{Fraktion A} \linebreak Papier, Pappe	& \textbf{Fraktion B} \linebreak Kunststoffe, etc.	& \textbf{Fraktion C} \linebreak Feinmüll & \textbf{Fraktion D} \linebreak Bioabfall\\
		\hline
		Kohlenstoff C  $ [TS\%] $ & 30-40&40&54&19&31\\
		Wasserstoff H  $ [TS\%] $ &4-5&6&8&3&4\\
		Stickstoff N  $ [TS\%] $ &0,3-0,5&0,3&0,9&0,4&2,1\\
		Sauerstoff O  $ [TS\%] $ &17-30&37&17&17&20\\
		\hline
	\end{tabulary}
}
\end{table}
\FloatBarrier
{\footnotesize \textbf{Fraktion A:} Papier, Pappe \quad \quad \quad \hspace*{0.7mm} \textbf{Fraktion B:} Kunststoffe, Holz, Leder, Textilien, Verbunde}\\
{\footnotesize \textbf{Fraktion C:} Feinmüll < 8 mm \quad \quad \textbf{Fraktion D:} Mittelmüll 8 – 40 mm, Vegetabilien}\\ \\
Der Wassergehalt wurde aus dem errechneten Trockensubstanzanteil berechnet (siehe Gleichung (\ref{eq:TS}) bis (\ref{eq:WG})).
Die Feuchtigkeit in der untersuchten Müllprobe kann praktisch nur in Papier, den Fasern und ähnlichem enthalten sein. Kunststoffe binden Feuchtigkeit praktisch nur als Anhaftung an ihrer Oberfläche \cite[S.3]{LLA_Abfallanalyse}. Der Müll fühlte sich trocken an und haftete auch nicht feuchtigkeitsbedingt zusammen.
Eine allgemeine Hausmüllprobe besitzt einen Wassergehalt von \mbox{$\approx 35\%$ } \mbox{(siehe Abb. \ref{fig:ersatzbrennstoffe})}. Vergleicht man die untersuchte Probe damit, erscheint der berechnete Wert für den Wassergehalt als nicht plausibel. \\
Da jedoch auch für Müllfraktionen wie Steine, Kunststoffe, Keramik, Verbundstoffe, etc. kein Wassergehalt in der \mbox{Quelle (siehe Abb. \ref{fig:ersatzbrennstoffe})} angegeben ist, kann man darauf schließen, dass der Wassergehalt nur bei entsprechend hygroskopischen bzw. schon stark feuchten bis nassen Müllfraktionen berücksichtigt wird. Der Wassergehalt der Müllprobe fällt mit $\approx \SI{3}{\percent}$ als sehr gering aus und kann somit als nahezu "`trocken"' angenommen werden.\\
Der Unterschied im Wassergehalt zwischen der originalen Probe und der für die TIC-Analyse behandelten Probe lässt sich auf die zugegebene Phosphorsäure bzw. Natronlauge zurückführen. Dieser Wassergehalt gibt keine offensichtlich sinnvolle Information preis.\\
Der Glühverlust wird im Folgenden indirekt über den Inertstoffgehalt verglichen. Laut der Quelle (siehe Abb. \ref{fig:ersatzbrennstoffe}) besitzt der allgemeine Haushaltsmüll einen Inertstoffgehalt von $\approx \SI{27}{\mpercent}$. 
Aschegehalte des Hausmülls schwankten Mitte der 1990er Jahre zwischen \SI{25}{\mpercent} und \SI{35}{\mpercent} \cite{scholz2013}.
Das Papier (Illustrationspapier) aus einem Magazin weist einen Glührückstand von ca. \SI{15}{\mpercent} auf \cite{roempppap}. Einige Kunststoffe können theoretisch sogar rückstandsfrei verbrennen. Es besteht also ein enger Bezug des Glührückstands zum Füllstoffgehalt. Dazu wären jedoch genauere Analysen zu den enthaltenen Kunststoffen nötig.\\
Im Vergleich dazu erscheint der Inertstoffgehalt der untersuchten Müllprobe mit $\approx \SI{73}{\percent}$ als zu hoch bzw. der Glühverlust im Vergleich als sehr niedrig. Die Vermutung, dass die optisch ascheähnlichen Bestandteile tatsächlich inert sind, könnte man somit begründen.
%Start
\begin{figure}[h!]
	\centering
	\includegraphics[width=0.90\textwidth]{img/ThermischeVerfahren}
	\caption[Darstellung eines Hausmülls durch unterschiedliche Stoffgruppen und deren Aufteilung auf die Komponenten Wasser, Inertstoff, Kunststoff und sonstigen organischen Komponenten]{Darstellung eines Hausmülls durch unterschiedliche Stoffgruppen und deren Aufteilung auf die Komponenten Wasser, Inertstoff, Kunststoff und sonstigen organischen Komponenten aus \cite[S.23]{scholz2013}}
	\label{fig:ersatzbrennstoffe}
\end{figure}
\FloatBarrier
%Ende

 

%\begin{itemize}	
%	\item Plausibilität TS\\ 
%	\textcolor{red}{Damit war eher die Trockensubstanz gemeint, wie bei den Berechnungen  unter dem Abschnitt vorher}\\
%	\textit{Der Schwefelgehalt des Abfalls setzt sich aus dem Organischen Schwefelgehalt (TOS) und dem Anorganischen Schwefelgehalt(TIS) zusammen. Aus dem Artikel \cite{Schwarzboeck2018} lässt sich ein Referenzwert für den TOS von etwa 8 bis \SI{10}{\gram\per\kilogram} bei Asche- Und Wasserfreiheit ableiten. Das entspricht \SI{0,64}{\mpercent} bis etwa \SI{1}{\mpercent}(siehe Rechnung weiter unten vielleicht noch in den Anhang)
%		Schwefelgehalte des Hausmülls schwankten mitte der 1990ger Jahre zwischen \SI{0,3}{\mpercent} und \SI{0,5}{\mpercent} \cite{scholz2013}.}	
%\end{itemize}

%	Der im Experiment ermittelte Glührückstand von rund 75\% ist deutlich zu hoch. In der Literatur ist nirgends ein so hoher mineralischer Füllstoffanteil, Glührückstand oder Aschegehalt beschrieben. Es sind 5 wichtige Fehlerquellen zu beachten. Der Müll könnte durch einen beachtlichen mineralischen Feststoffanteil belastet sein. Die Veraschung könnte unvollständig sein, wodurch verbliebene Organische Anteile als anorganisch interpretiert würden. Die Asche könnte bei den angewendeten Temperaturen ihre chemische Zusammensetzung in Anwesenheit von Umgebungsluft verändert haben. Die Asche könnte Feuchtigkeit aus der Umgebungsluft aufgrund ihrer hygroskopischen Eigenschaften angezogen haben. Es könnten Messfehler passiert sein. Bei Vertauschung der Tiegel könnten die Massen falsch subtrahiert worden sein. Dann hätten die Werte aber stark von einander abweichen müssen, weil sich der Fehler für die eine Masse negativ und für die andere Masse positiv ausgewirkt hätte. Letzterer Fehler kann damit ausgeschlossen werden. 
%	In Anbetracht der vielen möglichen Einflussfaktoren muss der ermittelte Wert kritisch betrachtet und vorsichtig verwendet werden. Er kann aber nicht als falsch ausgeschlossen werden.}

%Tabelle START
\vspace*{-3.5mm}
\renewcommand{\arraystretch}{1.2}
\begin{table}[h!]
	\centering
	\caption[Tabellenausschnitt mit Heizwerten üblicher Brennstoffe]{Tabellenausschnitt mit Heizwerten üblicher Brennstoffe aus \cite{S.Furkus.}}
	\label{tab:heizwerte}
	%\resizebox{10cm}{!}{
	\begin{tabulary}{\textwidth}{C|CC}
		\hline
		\textbf{Energieträger} & \textbf{Heizwert $\boldsymbol{\left[\si{\kWh\per\kg}\right]}$} & \textbf{Brennwert $\boldsymbol{\left[\si{\kWh\per\kg}\right]}$} \\ 
		\hline
		Stadtgas			&	6,8...8,7	&	7,7...9,8\\
		Benzin				&	12,11	&	13,06\\
		Waldfrisches Holz	&	1,90	&	k.A.\\
		Hackschnitzel		&	3,5...4,0	&	3,8...4,3\\
		Papier 				& 	4,17 	&	k.A. \\	
		Hausmüll			&	2,78	&	k.A.\\
		\hline
	\end{tabulary}
	%}
\end{table}
\FloatBarrier
\vspace*{-2.5mm}
%Tabelle Ende

Der Brennwert ist, richtiger Weise, höher als der Heizwert, da die im Wasserdampf gebundene Energie mit folgender Kondensation betrachtet wird. Ein Gemisch  welches zu gleichen Teilen aus Haus- und Papiermüll besteht hat etwa einen Heizwert von \SI{3,5}{\kilo\watt\hour\per\kilogram} oder \SI{12,6}{\mega\joule\per\kilogram} (siehe Tab. \ref{tab:heizwerte}). Mit \SI{1,65}{\kWh\per\kg} ist der berechnete Brennwert etwa halb so groß. Verglichen mit üblichen Brennstoffen aus \mbox{Tab. \ref{tab:heizwerte}} zeigt sich die Müllprobe als Ersatzbrennstoff mit \SI{1,65}{\kWh\per\kg} schlechter als die angegebenen Brennwerte für Benzin, Stadtgas und Hackschnitzel. Das erscheint sinnvoll durch den hohen Inertstoffgehalt, da dadurch pro Kilogramm weniger Energie durch Verbrennung frei werden kann. Für weitere Vergleiche wird Bezug auf den Heizwert genommen, da hierfür mehr Angaben aus Tab. \ref{tab:heizwerte} zu entnehmen sind.\\ 
Der Heizwert der Probe siedelt sich mit $\approx \SI{5}{\mega \joule \per \kg}$ bzw. $\approx \SI{1,5}{\kWh\per\kg}$ vergleichsweise zwischen die Müllfraktionen von Feinmüll mit $\approx \SI{4,29}{\mega \joule \per \kg}$ und Holz, Vegetabilien mit $\approx \SI{5,83}{\mega \joule \per \kg}$ an (siehe Tab. \ref{fig:ersatzbrennstoffe}). Aus rein optischer Einschätzung scheint lediglich der Feinmüll als plausibel, da Holz bzw. Vegetabilien nicht vorrangig zu erkennen sind. Die untersuchte Müllprobe liegt somit unter dem Heizwert für allgemeinen Hausmüll mit $\approx \SI{8,00}{\mega \joule \per \kg}$ (siehe Tab. \ref{fig:ersatzbrennstoffe}). Das wiederum erscheint durch den hohen Inertstoffgehalt als sinnvoll, nimmt man den Messwert der Müllfraktion von Papier, Pappe, etc. $\approx \SI{11,96}{\mega \joule \per \kg}$ in die Betrachtung des Heizwertes als Bestandteil hinzu (siehe Tab. \ref{fig:ersatzbrennstoffe}). Aufgrund der unterschiedlichen Zusammensetzungen von Müllproben erscheint eine genauere Plausibilitätsbetrachtung als schwierig. \linebreak
Die Dimensionen dieser Müllprobe scheinen, was den Heizwert anbelangt, jedoch nicht unrealistisch zu sein.\\
Ausgehend von der Sinnhaftigkeit der erhobenen Daten lässt der Heizwert, ähnlich wie der Brennwert, ebenfalls mit Heizwerten von üblichen Brennstoffen der \mbox{Tabelle \ref{tab:heizwerte}} vergleichen. Die Probe schneidet im Vergleich, auch in Bezug auf weitere Energieträger, deutlich schlechter ab. Lediglich waldfrisches Holz, mit \SI{1,90}{\kWh\per\kg} Heizwert, ist an dieser Stelle annähernd vergleichbar. Somit wird auch durch den Heizwertvergleich deutlich, dass sich die Müllprobe selbst, vergleichsweise schlecht als Ersatzbrennstoff eignet. Begründung liegt auch hier wieder im hohen Inertstoffgehalt, welcher keinen energetischen Nutzen darstellt.
\newpage
Um energetisch hochwertigere Brenn- bzw. Heizwerte zu erzielen, könnte untersucht werden ob eine Reduzierung des Inertstoffgehaltes pro Kilogramm, durch feineres Sieben der Müllprobe, zielführend ist.\\ \\
Zusammenfassend lässt sich sagen, dass lediglich Aussagen zur qualitativen Beschaffenheit getroffen können, da die Zusammensetzung der Müllprobe nicht genauer bestimmt ist. Ein Widerspruch ist hier nicht zu erkennen, da die Brenn- bzw. Heizwertewerte zwar im Vergleich zu anderen Energieträgern gering ausfallen, aber immerhin in derselben Größenordnung liegen.\linebreak
Erwartungsgemäß ist der Heizwert geringer als der Brennwert da die latente Wärme des Wasserdampfes in diesem nicht eingerechnet wird.\\
%===========================================
%Optional Berechnung zum Schwefelgehalt.

%	Eine Beladung von 8 - 10g/kg mit Schwefel bei Asche und Wasserfreiheit
%	\begin{flalign}
%	\frac{10 g}{1000 g} &= 1\%\\
%	\frac{8 g}{1000 g} &= 0,8\%
%	\end{flalign}
%	Angleichung an reale Verhältnisse durch Annahme von maximal 20 ma\% Asche und 5\% Wasser.
%	\begin{flalign}
%	\frac{10 g}{1250 g} &= 0,8\%\\
%	\frac{8 g}{1250 g} &= 0,64\%
%	\end{flalign}
%=============================================

Betrachtet man nun den Chlorgehalt, so liegt dieser üblicher Weise zwischen etwa \SI{0,3}{\mpercent} und \SI{3}{\mpercent}  \cite{LLA_Abfallanalyse}. Chlorgehalte des Hausmülls schwankten Mitte der 1990ger Jahre zwischen \SI{0,4}{\mpercent} und \SI{1}{\mpercent} \cite[S.22, Tab. 2.1]{scholz2013}. Der hier gemessene Chlorgehalt von \SI{0,17}{\mpercent} liegt damit unterhalb der Erwartung. Es ist zu beachten, dass der hier gemessene Chlorgehalt lediglich organisch gebundenes Chlor beschreibt. Anorganische Chloride wie etwa das Natriumchlorid geben bei den verwendeten Temperaturen kein Chlor ab. Es ist unklar ob die Literaturwerte sich nur auf organisches Chlor oder auch auf anorganisches Chlor beziehen. Nimmt man an, dass das Verhältnis der leicht zu zersetzenden Verbindungen zum schwer zu zersetzenden chlorhaltigen Verbindungen $1:1$ ist, so passt der Messwert zum Minimalwert der Literaturquellen \cite{LLA_Abfallanalyse}, \cite{scholz2013}. Dann läge der Gesamtchlorgehalt bei vielleicht \SI{3}{\gram\per\kilogram} und rund \SI{0,3}{\mpercent}. Bei der Bestückung der kleinen Probenträger mit feinen Anteilen der Müllprobe könnte auch zu wenig Kunststoff übertragen worden sein. Der geringe Chloranteil der Papierfraktion erklärt die Abweichung zum Literaturwert welcher sich auf den heterogenen Hausmüll bezieht. Papier sollte weniger als \SI{0,5}{\gram\per\kilogram} AOX (Adsorbierbare organisch gebundene Halogene) aufweisen \cite{lexikon}. Nur ein Bruchteil des AOX-Gehaltes entfällt dabei auf Chlor. Es lag kein reines Papier vor. Die Probe wies aber auch zu wenig Kunststoff für einen hausmülltypischen Chlorgehalt auf. Der ermittelte Messwert liegt somit zwischen den für die beiden Extremfälle zu erwartenden Konzentrationen. Somit kann der ermittelte Chlorgehalt auch nicht als falsch ausgeschlossen werden.


