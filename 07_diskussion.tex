\chapter{Diskussion}
\label{sec:diskussion}

Plausibilität Heizwert

Plausibilität Brennwert

Plausibilität TS

Plausibilität Wassergehalt

Plausibilität Glühverlust

Plausibilität Carbonatgehalt

Plausibilität TIC


Die wichtigste anorganische Kohlenstoffquelle in der Müllprobe sind vermutlich Carbonate.
Carbonate sind wichtige Füll- und Zuschlagsstoffe für Papier, Pappe und diverse Kunststoffe. Herkömmliches Papier kann dabei einen Füllstoffanteil von bis zu 30 \% aufweisen. In Kunststoffen dient Calciumcarbonat unter anderem zur Steigerung der Schlagfestigkeit. 

Die unbekannte Zusammensetzung der Müllprobe und die uneinheitliche Anwendung von Zuschlag- und Füllstoffen lassen kaum einen Vergleich mit anderen Proben zu. 



Wikipedia::
Wichtige Füllstoffe von thermoplastischen Kunststoffen sind:

Glasfasern, Glaskugeln und Glasbruch
mineralische Füllstoffe wie Calciumcarbonat und Talkum
Kohlenstofffasern (Kurzfasern)
Ruße

Bei der Papierherstellung werden vor allem Silikate, meist Kaolin, eine weiße Porzellanerde, als Füllstoff verwendet. Kaolin macht das Papier undurchsichtiger (opaker), weißer und erhöht die Rohdichte. Auch gibt der Füllstoff dem Papier eine glattere Oberfläche, da er die Hohlräume zwischen den Fasern auffüllt. Papier kann, abhängig von der Sorte, bis zu 30 % Füllstoff enthalten.

Als Füllstoff werden oft Carbonate verwendet, meistens Kreide, aber ebenso Sulfate wie Gips oder Oxide, beispielsweise Titandioxid. Bariumsulfat kommt als Füllstoff zur Herstellung von Barytpapier in Betracht, das dadurch auffallend schwer ist.[3] 

Heizwerte Plausibel siehe Quelle\\

In der Endkonsequenz könnte jeder Carbonatgehalt durch Füllstoffe erklärt werden. Ein weiterer Einflussfaktor sind Verunreinigungen. Teilweise werden anfallende Abfälle nicht korrekt getrennt. Ein wahrscheinliches Szenario wäre die Entsorgung nicht vollkommen restentleerter Kalksäcke. Die Säcke bestehen aus einer Kunststoffschicht und einer äußeren Papierhülle. Der enthaltene Kalk verfälscht die Messergebnisse. Ebenso könnte Kehricht im untersuchten Müll entsorgt worden sein.


Quelle in der Bibliothek suchen

Plausibilität TC

Plausibilität Cl-Gehalt\\

%\begin{tikzpicture}
%\begin{axis}[
%xbar=1pt,% space of 0pt between adjacent bars
%bar width=11,
%width=15cm,
%height=11cm,
%%minor y tick num=4,
%xmax=40,xmin=-40,
%x tick label style={/pgf/number format/.cd,%
%	scaled x ticks = false,
%	set decimal separator={,},
%	fixed},
%symbolic y coords={Zn,Cl,Mn,Uub,As},
%ytick=data,
%xtick={-30.5,-20.5,-10.5,0,10.5,20.5,30.5},
%grid=major,
%%enlargelimits=0.15,
%]
%\addplot coordinates {
%	(-10.3,Mn) (15.4,Cl) (5,Zn) (24,Uub) (30,As)
%};
%\addplot coordinates {
%	(-3,Mn) (5,Cl) (15,Zn) (20,Uub) (35,As)
%};
%\addplot coordinates {
%	(-8,Mn) (-19,Cl) (20,Zn) (30,Uub) (5,As)
%};
%
%\end{axis}
%\end{tikzpicture}

\nocite{ersatzbrennstoffe,PolymerServiceGmbHMerseburg.13.08.2019,Wikipedia.21.11.2019,Skript}