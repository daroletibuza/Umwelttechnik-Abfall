\chapter{Diskussion}
\label{sec:diskussion}

In diesem Abschnitt wird die Plausibilität der erhobenen Daten zur Analyse der Müllprobe 2 überprüft und diskutiert.\\ \\
Beginnend mit dem Carbonatgehalt bzw. dem TIC-Gehalt werden diese als wichtige Füll- und Zuschlagsstoffe für Papier, Pappe und diverse Kunststoffe eingesetzt. Für die untersuchte Müllprobe 2 fällt mit einem mittleren Carbonatgehalt von $\SI{8.9}{\mpercent}$ (siehe Abschnitt \ref{sec:tic}) dieser im Vergleich zu Carbonatgehalten für Papier mit bis zu $\SI{30}{\mpercent}$ \cite{Wikipedia.21.11.2019} und für Kunststoffe von $10 - \SI{50}{\mpercent}$ \cite{PolymerServiceGmbHMerseburg.13.08.2019} gering aus. Da aus der optischen Betrachtung des Mülls nicht nur Papier- und Kunststoffreste zu erkennen sind, sondern auch fasrige und veraschte Komponenten, ist es durchaus plausibel, dass der Carbonatgehalt entsprechend kleiner ausfällt. \\
In diesem Versuch wurde der Carbonatgehalt für zwei Stichproben bestimmt, welche von einander eine Abweichung von 3\% aufweisen, was sich auf die inhomogene Zusammensetzung der Müllprobe selbst zurück führen lassen kann.\linebreak
Die Abweichungen bezüglich der Kalibrierkurve könnten sich ebenfalls auf diese Inhomogenität zurückführen lassen, jedoch wäre an dieser Stelle entweder weitere Kalibrierpunkte, sowie weitere Messungen mit Müllproben sinnvoll, um repräsentative Aussagen treffen zu können. Nähere Infos zur Fehlerbetrachtung finden sich unter Abschnitt \ref{sec:fehler}.\\

Der TC-Gehalt liegt laut den Messergebnissen bei rund $\approx 25\%$ bzw. aus der Summe zwischen TIC und TOC bei  $\approx 23\%$ (siehe Tabelle \ref{tab:tc}). Aufgrund der optischen Zusammensetzung des Mülls teilweise enthaltenen Papier, Pappe und Kunststoffschnipseln sowie den Textilfasern und ascheähnlichen Resten könnte der Kohlenstoffgehalt plausibel sein. Verglichen mit Tabelle \ref{tab:tc_vergleich} \cite[S.13]{HansGunterRamke.} scheint sich die Plausibilität mit den Daten für Feinmüll und gemischt mit Pappe, Papier und Kunststoffen zu decken.\\

TABELLLLLLE von \cite{HansGunterRamke.}

\vspace*{5mm}

\begin{itemize}
	\item Plausibilität TC \cite{HansGunterRamke.}
	\item 
	\item Plausibilität TS
	\item Plausibilität Wassergehalt
	\item Plausibilität Glühverlust
	\item Plausibilität Heizwert \cite{S.Furkus.}
	\item Plausibilität Brennwert \cite{S.Furkus.}
\end{itemize}



\newpage

Die wichtigste anorganische Kohlenstoffquelle in der Müllprobe sind vermutlich Carbonate.
Carbonate sind wichtige Füll- und Zuschlagsstoffe für Papier, Pappe und diverse Kunststoffe. Herkömmliches Papier kann dabei einen Füllstoffanteil von bis zu 30 \% aufweisen. In Kunststoffen dient Calciumcarbonat unter anderem zur Steigerung der Schlagfestigkeit. 

Die unbekannte Zusammensetzung der Müllprobe und die uneinheitliche Anwendung von Zuschlag- und Füllstoffen lassen kaum einen Vergleich mit anderen Proben zu. 



Wikipedia::
Wichtige Füllstoffe von thermoplastischen Kunststoffen sind:

Glasfasern, Glaskugeln und Glasbruch
mineralische Füllstoffe wie Calciumcarbonat und Talkum
Kohlenstofffasern (Kurzfasern)
Ruße

Bei der Papierherstellung werden vor allem Silikate, meist Kaolin, eine weiße Porzellanerde, als Füllstoff verwendet. Kaolin macht das Papier undurchsichtiger (opaker), weißer und erhöht die Rohdichte. Auch gibt der Füllstoff dem Papier eine glattere Oberfläche, da er die Hohlräume zwischen den Fasern auffüllt. Papier kann, abhängig von der Sorte, bis zu 30 % Füllstoff enthalten.

Als Füllstoff werden oft Carbonate verwendet, meistens Kreide, aber ebenso Sulfate wie Gips oder Oxide, beispielsweise Titandioxid. Bariumsulfat kommt als Füllstoff zur Herstellung von Barytpapier in Betracht, das dadurch auffallend schwer ist.[3] 

Heizwerte Plausibel siehe Quelle\\

In der Endkonsequenz könnte jeder Carbonatgehalt durch Füllstoffe erklärt werden. Ein weiterer Einflussfaktor sind Verunreinigungen. Teilweise werden anfallende Abfälle nicht korrekt getrennt. Ein wahrscheinliches Szenario wäre die Entsorgung nicht vollkommen restentleerter Kalksäcke. Die Säcke bestehen aus einer Kunststoffschicht und einer äußeren Papierhülle. Der enthaltene Kalk verfälscht die Messergebnisse. Ebenso könnte Kehricht im untersuchten Müll entsorgt worden sein.
\cite{domininghausKunststoffeUndIhre1998}
\cite{roempppap}
Plausibilität Cl-Gehalt\\

Plausiilität Glührückstand

Illustrationspapier (= Magazinpapier): holzhaltiges, ungestrichenes Druckpapier, satiniert, mit einem Glührückstand (Aschegehalt ca. 15 Gewichts- \%) und Flächengewichten zwischen 55 und 80g/m2. 

%\begin{tikzpicture}
%\begin{axis}[
%xbar=1pt,% space of 0pt between adjacent bars
%bar width=11,
%width=15cm,
%height=11cm,
%%minor y tick num=4,
%xmax=40,xmin=-40,
%x tick label style={/pgf/number format/.cd,%
%	scaled x ticks = false,
%	set decimal separator={,},
%	fixed},
%symbolic y coords={Zn,Cl,Mn,Uub,As},
%ytick=data,
%xtick={-30.5,-20.5,-10.5,0,10.5,20.5,30.5},
%grid=major,
%%enlargelimits=0.15,
%]
%\addplot coordinates {
%	(-10.3,Mn) (15.4,Cl) (5,Zn) (24,Uub) (30,As)
%};
%\addplot coordinates {
%	(-3,Mn) (5,Cl) (15,Zn) (20,Uub) (35,As)
%};
%\addplot coordinates {
%	(-8,Mn) (-19,Cl) (20,Zn) (30,Uub) (5,As)
%};
%
%\end{axis}
%\end{tikzpicture}

